\documentclass[10pt]{report}
\usepackage[margin=1.25in]{geometry}
\usepackage[titletoc]{appendix}
\usepackage{verbatim}
\usepackage{graphicx}
\usepackage{xcolor}
\usepackage{dirtree}
% The following is a dummy icon command
\newcommand\myicon[1]{{\color{#1}\rule{2ex}{2ex}}}
% If you have actual icon images, use \includegraphics to include them
% If you are generating them, put in the appropriate code for them here
% now we make a command for a folder/file which inserts the icon and its label
% adjust this as needed. If you only have 2 icons, then you could create
% a \myfile and \myfolder command with the icon fixed.
\newcommand{\myfolder}[2]{\myicon{#1}\ {#2}}

\usepackage{hyperref}
\hypersetup{
    bookmarks=true,         % show bookmarks bar?
    unicode=false,          % non-Latin characters in Acrobat’s bookmarks
    pdftoolbar=true,        % show Acrobat’s toolbar?
    pdfmenubar=true,        % show Acrobat’s menu?
    pdffitwindow=false,     % window fit to page when opened
    pdfstartview={FitH},    % fits the width of the page to the window
    pdftitle={COAT User's Guide},    % title
    pdfauthor={Eric S. Maddy},     % author
    pdfsubject={Manual},   % subject of the document
    pdfcreator={Eric S. Maddy},   % creator of the document
    pdfkeywords={Manual} {COAT} {JCSDA}, % list of keywords
    pdfnewwindow=true,      % links in new window
    colorlinks=true,       % false: boxed links; true: colored links
    linkcolor=blue,          % color of internal links (change box color with linkbordercolor)
    citecolor=green,        % color of links to bibliography
    filecolor=magenta,      % color of file links
    urlcolor=cyan           % color of external links
}

\author{Eric S. Maddy \\ 
\\\vspace{6pt} Riverside Technology, inc. at NESDIS/JCSDA, \\ NCWCP College Park, MD 20740} \title{{COAT}: The Community Observation Assessment Tool} \date{\today}

\begin{document}
\maketitle
\pagenumbering{Roman}
\tableofcontents
%\newpage
%\listoffigures
%\newpage
%\listoftables
\newpage
\pagenumbering{arabic}
\chapter{Introduction}
\section{Getting Started}
\subsection{Checking out the COAT from orbit machines and SVN}
To checkout the COAT from the SVN on orbit machines (NESDIS/STAR network at NCWCP): 
\begin{verbatim} 
 $ svn checkout \
  file:///data/home001/dxu/SVN_REPOSITORY/observation_assessment_tool
\end{verbatim}
If you do not have an orbit account (STAR local account) and you have a NOAA NEMS email account, you can request one using the STAR IT helpdesk:
  \url{https://www.star.nesdis.noaa.gov/intranet/helpdesk/index.php}.
You'll need to ask for an orbit account. You'll also need an orbit account with Fortran and C compilers (ask Kevin G.: orbit272 or orbit273 might be an option).

\subsection{Getting the COAT from S4}
If you have an account on S4.  The COAT is available on S4-gateway: 
\begin{verbatim}
  $ cd {your-installation-path}
  $ mkdir COAT
  $ cd COAT
  $ cp /home/emaddy/COAT/coat.tar ./
  $ tar -xvf coat.tar
\end{verbatim}
\subsection{Software required to run and compile COAT routines}
\begin{itemize}
  \item Fortran compiler: ifort, gfortran, g95, xlf, pgi
  \item C compiler: icc, gcc, xlc, pgcc 
  \item \LaTeX~and\TeX~packages including pdf\LaTeX
  \item IDL Version 7 or above
\end{itemize}
\newpage
\subsection{COAT directory structure}
\dirtree{%
.1 \myfolder{darkgray}{COAT}.
.2 \myfolder{cyan}{bin}.
.2 \myfolder{cyan}{config}.
.2 \myfolder{cyan}{data}.
.3 \myfolder{gray}{bias\_coeffs}.
.3 \myfolder{gray}{crtm\_coeffs}.
.3 \myfolder{gray}{regr\_coeffs}.
.3 \myfolder{gray}{static}.
.3 \myfolder{gray}{TestbedData}.
.4 \myfolder{lightgray}{edr}.
.4 \myfolder{lightgray}{edrd}.
.4 \myfolder{lightgray}{fmsdr}.
.4 \myfolder{lightgray}{fmsdr\_sim}.
.4 \myfolder{lightgray}{nwp}.
.2 \myfolder{cyan}{document}.
.2 \myfolder{cyan}{images}.
.2 \myfolder{cyan}{log}.
.2 \myfolder{cyan}{scripts}.
.3 \myfolder{gray}{config}.
.4 \myfolder{lightgray}{template}.
.3 \myfolder{gray}{lists}.
.2 \myfolder{cyan}{src}.
.3 \myfolder{gray}{bufr}.
.4 \myfolder{lightgray}{amsr2\_bufr}.
.4 \myfolder{lightgray}{amv\_bufr}.
.4 \myfolder{lightgray}{bufrlib}.
.4 \myfolder{lightgray}{gmi\_bufr}.
.4 \myfolder{lightgray}{ssmis\_bufr}.
.3 \myfolder{gray}{collocation}.
.3 \myfolder{gray}{crtm}.
.4 \myfolder{lightgray}{REL-2.1.3...}.
.3 \myfolder{gray}{fwd}.
.3 \myfolder{gray}{idl}.
.3 \myfolder{gray}{lib}.
.3 \myfolder{gray}{readers}.
.3 \myfolder{gray}{regr\_retr}.
.3 \myfolder{gray}{scenedump}.
}
\newpage
\subsection{Compiling CRTM and COAT routines}
CRTM REL-2.1.3 is included in the COAT distribution.  The first step in compiling COAT routines is to 
configure compiler settings and to compile the CRTM.   
\begin{verbatim}
  # make CRTM
  $ cd COAT/src/crtm/REL-2.1.3/
  $ cd configure/
  $ source ifort.setup
  $ cd ..
  $ make
  $ make install
\end{verbatim}
Once the CRTM library is successfully compiled and the library/module files are installed, the next 
step is to compile the COAT library routines.  The COAT BUFR library routines ({\tt src/bufr/bufrlib}) 
require Fortran and C compilers.  You'll need to make sure that the C compiler (variable {\tt CC}) is 
defined and the architecture (x86 or x86\_64) is compatible with the Fortran compiler used 
for the compilation of the CRTM.
\begin{verbatim}
  # make all COAT routines (output exes are in COAT/bin/)
  $ cd ../../
  $ make 
\end{verbatim}

\chapter{IDL COAT basics}
\section{IDL COAT input - user defined parameters}
The COAT user defined parameters are located in: 
\begin{itemize}
  \item {\tt configSensorParam.pro}
  \item {\tt config/configParam\_\{SAT\}\_\{SENSOR\}.pro}
  \item {\tt main\_AT.pro}
\end{itemize}
Not all {\tt configParam\_\{SAT\}\_\{SENSOR\}.pro} files have correct values for 
all instrument dependent parameters.  The user is required to properly set the number of channels,
channel frequencies, latitudes and longitudes of valid data, MIN and MAX expected BTs, etc.  Below 
is a list of parameters that are set for GPM GMI.  
\begin{verbatim}
   ; Set map range to plot
   paramStruct.MIN_LAT = -90
   paramStruct.MAX_LAT = 90
   paramStruct.MIN_LON = -180
   paramStruct.MAX_LON = 180

   ; Set Max profile number and channel number
   paramStruct.MAX_FOV = 50000L
   paramStruct.MAX_CHAN = 13L 

   ; Radiance files (obs + sim)
   paramStruct.radListFile1 = 'meas.list'
   paramStruct.radListFile2 = 'fwd.list'
   ; Scene file (collocated NWP profiles to observations) can 
   ; be scene or scenedump format
   paramStruct.sceneListFile = 'scene.list'

   ; Bias and standard deviation files
   paramStruct.biasListFile = 'bias.list'
   paramStruct.stddevListFile = 'stddev.list'

   ; location of the clw regression coefficients for clear tests
   paramStruct.clwCoeffPath = './data/regr_coeffs/'
   ; string array of channel numbers
   paramStruct.chanNumArr =   $
       ['1','2','3','4','5','6','7','8','9','10',   $
       '11','12','13']
   ; string array of channel frequencies in GHz with polarization (not necc.)
   paramStruct.chanInfoArr  =  $
       ['10.65V','10.65H','18.7V','18.7H','23.8V',$
        '36.5V','36.5H','89.0V','89.0H','165.5V','165.5H',$
        '183.31V','183.31H']
   ; minimum BT values to assume are good data
   paramStruct.minBT_Values =  FLTARR(13) + 20.
   ; maximum BT values to assume are good data
   paramStruct.maxBT_Values =  FLTARR(13) + 320.
   ; maximum ABS(delta.BT = OBS-SIM) values to assume are good data
   paramStruct.maxDTb_Values =  FLTARR(13) + 20.
   paramStruct.sensorName = 'GPM_GMI'
   paramStruct.sensorID= 'gmi'
   ; array of channels to plot - IDL 0-based index 
   paramStruct.chPlotArr = INDGEN(13)
   ; data date
   paramStruct.date = '2014-06-23'
   ; Image output dir
   paramStruct.imageDir     = './images/GPM/'

   ;---histogram trim options for clear
   ; trimoption (+-ive : scale un-filtered histogram at 
   ; trimOption*std.dev of Obs-Sim, --ive: use trimValue for 
   ; ABS(Obs-Sim) trimming
   paramStruct.histogramOptions.trimOption  = INTARR(13) + 2
   ; trimValue (if timeOption < 0) use this value for trimming.
   paramStruct.histogramOptions.trimValue   = FLTARR(13) + 1.
   ; useforclear - if 0 - don't use channel for trimming "clear" cases.
   ;               if 1 - use channel for trimming of "clear" cases.
   paramStruct.histogramOptions.useforclear = [LONARR(13)]

\end{verbatim}
Other parameters that control filtering for clear, cloudy, and precipitating cases 
are defined in {\tt configSensorParam.pro}.  These parameters
are: 
\begin{verbatim}
  ; minimum CLW threshold (below CLW_THRESHOLD_MIN is defined as "clear")
  CLW_THRESHOLD_MIN : 0.05,  $
  ; max CLW threshold (between CLW_THRESHOLD_MIN and CLW_THRESHOLD_MAX, 
  ; the scene is defined as being "cloudy", 
  ; above CLW_THRESHOLD_MAX is defined as a "precipitating")
  CLW_THRESHOLD_MAX : 0.3,   $
  ; not currently used
  RWP_THRESHOLD     : 0.05,  $
  GWP_THRESHOLD     : 0.05,  $
\end{verbatim}

The final user defined parameters are in the main script to run the COAT in IDL - {\tt main\_AT.pro}.  
The parameters that can be selected by users are : 
\begin{itemize}
  \item {\tt PLOT\_UNFILTERED} - plot all cases
  \item {\tt PLOT\_CLRSKY} - plot ``clear'' cases only
  \item {\tt PLOT\_CLDSKY} - plot ``cloudy'' cases only
  \item {\tt PLOT\_PCPSKY} - plot ``precipitating'' cases only
\end{itemize}
A value of 1 produces .eps output for these types, while a value of 0 turns off .eps output for these scene types.  As discussed above, the thresholds for determining, clear, cloudy and precipitating are set in {\tt configSensorParam.pro}.

\section{IDL COAT}
See Section~\ref{sec:IDLCOATEx} for examples on how to run the IDL COAT on sample (or other) datasets.

\section{IDL COAT output}
The COAT produces a large number of \.eps files in {\tt configParam.imagedir} defined in\\ 
{\tt configParam\_\{SAT\}\_\{SENSOR\}.pro}. 
These files are then run through the IDL code \\ {\tt Create\_DataAssessmentReport.pro}
which creates \TeX~formatted files and then calls pdf\LaTeX~to produce 
two data assessment reports or DARs.  
\begin{itemize}
  \item {\tt GPM\_GMI\_DAR\_maps.pdf}: maps of observed BTs, simulated BTs and observed minus simulated BTs 
    for all channels, and lat/lon limits defined by the user in {\tt configParam*.pro}.
  \item {\tt GPM\_GMI\_DAR\_stratified.pdf}: table summaries of statistics for all and ``clear-sky'' filtered cases over 
    ocean as well as scatterplots of observed versus simulated BTs and observed minus simulated BTs as a function of 
    various parameters.
\end{itemize}

\chapter{Fortran COAT basics}
\section{Fortran Routines}
\begin{itemize}
  \item {\tt bin/AMVBUFR\_to\_AMVScene} - convert GDAS SATWND BUFR data to AMV scene format.
  \item {\tt bin/colocNWPwRad} - collocate NWP profiles to satellite observed BT latitude, longitude, viewing geometry/etc.
  \item {\tt bin/colocNWPsatwind} - collocate NWP wind fields to satellite observed AMV latitude, longitude and pressure.
  \item {\tt bin/fwd} - compute satellite simulated BTs from NWP MIRS scene file (e.g. from {\tt bin/colocNWPwRad}).
  \item {\tt bin/fwdlist} - compute satellite simulated BTs from a list of NWP MIRS scene files (e.g. from {\tt bin/colocNWPwRad}). 
  \item {\tt bin/sceneDump} - extract information from scene file for the COAT (see Section~\ref{SDex}).
\end{itemize}
\subsection{NWP Collocation and Forward Operator}
Section~\ref{sec:COATscripts} provides an overview on how to run NWP collocation and the CRTM forward operator.  This section provides a 
brief overview of the settings assumed in these steps.
\subsubsection{NWP Collocation namelist}\label{ssec:NWPColoc}
The NWP collocation namelist for {\tt colocNWPwRad} is build using a template 
file that is located in \\
 {\tt scripts/config/template/COLOC\_NWP\_Template.CONFIG}.
\begin{verbatim}
  $ cat COLOC_NWP_Template.CONFIG
  &ControlNWP
    RadFileList='OBS_RAD_LIST'
    atmNWPFilesList='NWP_ATM_LIST'
    sfcNWPFilesList='NWP_SFC_LIST'
    pathNWPout='EDR_PATH/DATE/'

    Topogr='../data/static/Topography/topography.bin_sgi'
    CovBkgFileAtm='../data/static/CovBkgStats/CovBkgMatrxTotAtm_all.dat'
    norbits2process=1000000
    LogFile='LOG_PATH/SENSOR_logFile.dat'
    nprofs2process=1000000
    sensor_id=SENSORID
    nwp_source=NWPID
    Coeff_Path='../data/crtm_coeffs/'
    BiasFile='../data/static/biasCorrec/biasCorrec_SAT.dat'
    TuningFile='../data/static/TuningData/TunParams_SAT_SENSOR.in'
  /
\end{verbatim}
Variables in all capitals (OBS\_RAD\_LIST, NWP\_ATM\_LIST, EDR\_PATH, DATE, 
LOG\_PATH, SENSOR, SAT, SENSORID, NWPID) are set in the script {\tt generate\_coloc\_config.bash} and imported from {\tt paths.setup} and 
{\tt instrument.setup}.  
There are three choices for NWPID, 
\begin{enumerate}
\item GDAS analysis
\item GFS 6 hour forecast
\item ECMWF analysis
\end{enumerate}
currently only the options for GFS and ECMWF are available.
\subsubsection{FWD namelist}
The FWD model namelist for {\tt fwd} and {\tt fwdlist} is built 
using a template file that is located in \\
 {\tt scripts/config/template/FWD\_Template.CONFIG}.
\begin{verbatim}
  $ cat FWD_Template.CONFIG
  &ContrlP_fwd
    CntrlConfig_fwd%GeophInputFile='EDR_COLOC_NWP_FILE'
    CntrlConfig_fwd%Coeff_Path='../data/crtm_coeffs/'
    CntrlConfig_fwd%InstrConfigFile= &
      'STATIC_DATA/InstrConfigInfo/InstrConfig_SATSENSOR.dat'
    CntrlConfig_fwd%OutputFile='FMSDR_SIM_RAD_FILE'
    CntrlConfig_fwd%nprofs2process=1000000
    CntrlConfig_fwd%SensorID=SENSORID
    CntrlConfig_fwd%ChanSel(1:NCHAN)=NCHAN*1
    CntrlConfig_fwd%iAddDeviceNoise=0
    CntrlConfig_fwd%iCldOffOrOn=0
    CntrlConfig_fwd%NoiseFile='nedt/NOISE_FILE'
    CntrlConfig_fwd%iPrintMonitor=0
    CntrlConfig_fwd%LogFile='LOG_PATH/LOG_FILE'
  /
\end{verbatim}
Similar to the NWP collocation namelist, variables in all capitals (EDR\_COLOC\_NWP\_FILE, STATIC\_DATA, FMSDR\_SIM\_RAD\_FILE, SENSORID, 
NCHAN, NOISE\_FILE, LOG\_PATH, and LOG\_FILE) are set in the script {\tt generate\_fwd\_config.bash} and imported from {\tt paths.setup} and 
{\tt instrument.setup}.  

Clouds can be turned on by modifying the template such that 
{\tt CntrlConfig\_fwd\%iCldOffOrOn=1}.

\subsection{BUFR converters}
\begin{verbatim}
  $ ls src/bufr/
   amv_bufr
   amsr2_bufr
   ssmis_bufr
   gmi_bufr
\end{verbatim}

\subsection{Scene Dump}
See Section~\ref{SDex}.

\chapter{Examples}
\section{Example: Running COAT scripts}\label{sec:COATscripts}
In the following sections we'll use GPM GMI observations (FMSDRs), collocated NWP from GFS (EDRs), and 
simulations from the EDRs (FMSDR\_SIMs) as an example of how to run the COAT and produce graphical output 
and data assessment reports or DARs.
To run the NWP collocation and forward model steps, you'll first need to have observed satellite instrument 
FMSDR data.  The data for this example is in \\ {\tt /data/data086/emaddy/dat/observation\_assesment\_tool/}:
\begin{verbatim}
   data/TestbedData/fmsdr/gpm_gmi/2014-06-23/
\end{verbatim}
however, you could also use your own {\it fmsdr} formatted data. 

General data paths for the COAT binaries are located in the {\tt scripts/} directory 
in the main COAT directory.  There are a number of .bash scripts and 
two .setup files which are used to set environment and runtime variables in the COAT.  For instance,
\begin{verbatim}
  $ cd scripts
  $ ls *.bash
   8 -rwxr-xr-x. 1 emaddy  4333 Aug 20 11:52 get_ecmwfwind.bash
   8 -rwxr-xr-x. 1 emaddy  5381 Aug 20 11:52 get_ecmwf.bash
   4 -rwxr-xr-x. 1 emaddy   763 Aug 20 11:52 generate_OBS_list.bash
   8 -rwxr-xr-x. 1 emaddy  4419 Aug 20 11:52 generate_fwd_config.bash
   4 -rwxr-xr-x. 1 emaddy  3086 Aug 20 11:52 generate_coloc_config.bash
   4 -rwxr-xr-x. 1 emaddy  1481 Aug 21 08:48 run_colocfwd.bash
  12 -rwxr-xr-x. 1 emaddy  9502 Aug 21 08:49 get_gfs_grib2wind.bash
  12 -rwxr-xr-x. 1 emaddy  8585 Aug 21 08:49 get_gfs_grib1wind.bash
  16 -rwxr-xr-x. 1 emaddy 14285 Aug 21 08:49 get_gfs_grib1.bash
  16 -rwxr-xr-x. 1 emaddy 14602 Aug 26 13:48 get_gfs_grib2.bash
  
  $ ls *.setup
   4 -rw-r--r--. 1 emaddy 305 Aug 20 11:52 instrument.setup
   4 -rw-r--r--. 1 emaddy 708 Aug 27 10:08 paths.setup
\end{verbatim}
{\tt paths.setup} defines general paths for the COAT scripts as well as the location of the COAT installation directory.  
\begin{verbatim}
  $ cat paths.setup
  #---basepath to the coat
  COAT_PATH=/data/data086/emaddy/dat/observation_assessment_tool

  #---sensor dependent data paths
  #---note that FMSDR, FMSDR_SIM, EDR, and NWP data paths need not be 
  #   in the same directory as the COAT.
  FMSDR_DATA_PATH=${COAT_PATH}/data/TestbedData/fmsdr/${SATSENSOR}
  FMSDR_SIM_DATA_PATH=${COAT_PATH}/data/TestbedData/fmsdr_sim/${SATSENSOR}
  EDR_DATA_PATH=${COAT_PATH}/data/TestbedData/edr/${SATSENSOR}
  BUFR_DATA_PATH=${COAT_PATH}/data/TestbedData/bufr/${SATSENSOR}
  STATIC_DATA=${COAT_PATH}/data/static
  CONFIG_PATH=${COAT_PATH}/scripts/config
  LIST_PATH=${COAT_PATH}/scripts/lists

  #---path to NWP binary data (converted from GRIB)
  NWP_DATA_PATH=${COAT_PATH}/data/TestbedData/nwp

  #---path to binary executables
  EXE_PATH=${COAT_PATH}/bin

  #---path for log files
  LOG_PATH=${COAT_PATH}/log
\end{verbatim}
In the above example, the FMSDR, FMSDR\_SIM, EDR, and NWP data paths are in the same directory as the COAT; 
however, this does not always need to be true.  The data files required by the COAT can take up large amounts 
of disk space and therefore a user may not have enough disk space to house or want all data files in the 
same directory as the COAT installation.  No worries-data for the COAT can be written 
anywhere locally accessable by the machine running the COAT.

Instrument parameters and setup variables required by the collocation as well as forward model 
steps are in the {\tt scripts/instrument.setup}.  Note that the {\tt instrument.setup} 
variable ``SENSORID'' can be determined from {\tt src/lib/Consts.f90}.
\begin{verbatim}
  $ cat instrument.setup
  #---sensor to run :: GCOMW1 AMSR2
  SAT=gcomw1
  SENSOR=amsr2
  SATSENSOR=${SAT}_${SENSOR}
  NCHAN=14
  SENSORID=15

  #---sensor to run :: F18 SSMI/S
  SAT=f18
  SENSOR=ssmis
  SATSENSOR=${SAT}_${SENSOR}
  NCHAN=24
  SENSORID=5

  #---sensor to run :: GPM GMI
  SAT=gpm
  SENSOR=gmi
  SATSENSOR=${SAT}_${SENSOR}
  NCHAN=13
  SENSORID=10
\end{verbatim}

\subsection{Dump ECMWF or GFS GRIB to COAT binaries}\label{ssec:GRIBdump}
This step can currently only be run on orbit machines.  You'll need to collocate NWP to 
observed FMSDR for the date of observations as well as the previous and following day.  For example,
for June 23, 2014, we would run:
\begin{verbatim}
  $ get_gfs_grib1.bash 20140622
  $ get_gfs_grib1.bash 20140623
  $ get_gfs_grib1.bash 20140624
\end{verbatim}
to dump GFS 6 hour forecast GRIB files into binary files, and: 
\begin{verbatim}
  $ get_ecmwf.bash 20140622
  $ get_ecmwf.bash 20140623
  $ get_ecmwf.bash 20140624
\end{verbatim}
to dump ECMWF analysis GRIB files into binary files.  After the GRIB files are dumped into binaries, 
we can then run the NWP collocation routines as well as the forward model simulations.

\subsection{Run NWP collocation and forward model calculations}\label{ssec:scriptNWP}
After the NWP data is dumped from GRIB1 or 2 to binary, the next steps are to run the NWP collocation 
and forward model calculation routines.  Assuming that the paths defined in {\tt paths.setup} are correct and 
the instrument parameters in {\tt instrument.setup} are correct, the bash script \\
{\tt scripts/run\_colocfwd.bash} will automate these next steps.  Using the same day as we did in 
Section~\ref{ssec:GRIBdump}, June 23, 2014, we run the collocation and forward model steps as follows:
\begin{verbatim}
  $ run_colocfwd.bash gpm_gmi 20140623
\end{verbatim}
{\tt run\_colocfwd.bash} will collocate GFS and ECMWF NWP profiles or EDRs to instrument observed FMSDRs 
in MIRS {\it scene} format (see {\tt src/lib/IO\_Scene.f90} and/or \\{\tt src/idl/mirs\_idl/io\_scene.pro}) 
and run the CRTM forward model to create simulated FMSDRs (FMSDR\_Sims) in MIRS {\it Measur} format 
(see {\tt src/lib/IO\_MeasurData.f90} and/or \\{\tt src/idl/mirs\_idl/io\_measur.pro}).

\section{Example: Running the IDL COAT using sample data}\label{sec:IDLCOATEx}

\subsection{Sample data on NESDIS/STAR orbit machines}\label{ssec:OrbitSamp}
On orbit machines, sample data for the COAT is located in:
\begin{verbatim}
  $ cd /data/data086/emaddy/dat/observation_assessment_tool/
  $ ls 
  data/TestbedData/edr/gpm_gmi/2014-06-23/
  data/TestbedData/fmsdr/gpm_gmi/2014-06-23/
  data/TestbedData/fmsdr_sim/gpm_gmi/2014-06-23/
\end{verbatim}
Copy data to your local COAT directory (e.g., {\tt COAT/data/Testbedata/}) and 
follow the instructions below.  Alternatively you could copy the FMSDR data from above to your 
local directory and follow the instructions in Sections~\ref{sec:COATscripts} through~\ref{ssec:scriptNWP}.

\subsection{Sample data on S4}\label{ssec:S4Samp}
On S4, sample data for the COAT is located in:
\begin{verbatim}
  /data/emaddy/TestbedData/edr/gpm_gmi/2014-06-23/
  /data/emaddy/TestbedData/fmsdr/gpm_gmi/2014-06-23/
  /data/emaddy/TestbedData/fmsdr_sim/gpm_gmi/2014-06-23/
\end{verbatim}
Copy data to your local COAT directory (e.g., {\tt COAT/data/Testbedata/}) and 
follow the instructions below.  On S4, you'll need to be on gateway so you can run IDL.  

\subsection{Next Steps in Running the COAT}\label{ssec:nextStep}
Once you have data copied locally, can see it from your machine, or have run the COAT scripts on your/sample data 
(e.g., Section~\ref{sec:COATscripts}), you'll need to make a list of files for 
the NWP EDRs, observed instrument FMSDRs, and simulated FMSDR\_SIM files.  The following example uses
NWP GFS scene files from colocNWPwRad and FMSDR\_SIM files resultant from calling the CRTM forward model with 
those inputs.
\begin{verbatim}
  $ cd COAT/
  $ ls -d -1 ${path-to-data}/edr/gpm_gmi/2014-06-23/*GFS* > scene.list
  $ ls -d -1 ${path-to-data}/fmsdr/gpm_gmi/2014-06-23/* > meas.list
  $ ls -d -1 ${path-to-data}/fmsdr_sim/gpm_gmi/2014-06-23/*GFS* > fwd.list
\end{verbatim}
Next, modify {\tt COAT/config/configParam\_GPM\_GMI.pro} to point to those list files\footnotemark.
  
\begin{verbatim}
   ; Radiance files (obs + sim)
   paramStruct.radListFile1 = 'meas.list'
   paramStruct.radListFile2 = 'fwd.list'
   ; Scene file
   paramStruct.sceneListFile = 'scene.list'
\end{verbatim}
\footnotetext{Note that the Radiances files in {\tt configParam*.pro} are in the same format.  A user could compare the output of two simulations with slightly modified input geophysical states by selecting:\\\texttt{paramStruct.radListFile1 = 'fwd1.list'\\   paramStruct.radListFile2 = 'fwd2.list',\\} where {\tt fwd1.list} is a list of control FWD simulation files and {\tt fwd2.list} is a list of experiment FWD simulation files with slightly modified input geophysical states.}

The full path to the files and list files has to be specified in the {\tt configParam\_*.pro} file.  
You'll also need to set the date as well as output directories for 
images (NOTE: the image directory must exist when running next steps) : 
\begin{verbatim}
  configParam.date = '2014-06-23'
  configParam.imagedir = 'images/gpm_gmi/'
\end{verbatim} 
To run the COAT, in IDL, First cd into the COAT directory, make the output directory for the images, and then start IDL: 
\begin{verbatim}
  $ cd COAT
  $ mkdir images/gpm_gmi
  $ idl
  IDL--> !PATH = !PATH + ':src/idl/coyote/'
  IDL-->.r main_AT
  Choose instrument:
   1 : NOAA-18/AMSUA&MHS
   2 : NOAA-19/AMSUA&MHS
   3 : MetOp-A/AMSUA&MHS
   4 : MetOp-B/AMSUA/MHS
   5 : F16/SSMIS
   6 : F17/SSMIS
   7 : F18/SSMIS
   8 : NPP/ATMS
   9 : AQUA/AMSRE
  10 : GCOMW1/AMSR2
  11 : FY3/MWRI
  12 : FY3/MWHS/MWTS
  13 : TRMM/TMI
  14 : GPM/GMI
  15 : MT/MADRAS
  16 : MT/SAPHIR
  17 : WindSat
  18 : SATWND AMV
  : 14
  Choose CLW data source
  1 - CLW algorithm
  2 - CLW from NWP
  3 - CLW from both
  : 2  ; for now 
  Choose Scene data source
  0 - New Scene dump file
  1 - Old Scene file
  : 1
  Read data again?
  1 - YES
  2 - NO, to reform data
  3 - NO, to plot data
  4 - NO, to create data assessment report (must have run steps 1-3 first)
  : 1
\end{verbatim}
\section{Example: Running {\tt sceneDump} to speed reading of {\it scene} EDR files in COAT}\label{SDex}
{\tt bin/sceneDump} (or source {\tt src/scenedump/sceneDump.f90}) is a Fortran 90 executable 
that reads an EDR file in {\it scene} format and dumps the quantities required by the COAT
to a smaller binary for faster reading.  The code is namelist driven, with the user providing
a list of scene files ``SceneFileList,'' the location of the output directory ``pathout,'' and 
the name of a ``LogFile.''
\begin{verbatim}
 $ cat scene_dump.nl
  &ControlSD
    ! list of NWP GFS or ECMWF scene files from colocNWPwRad
    SceneFileList='scene.list'  
    pathout='data/TestbedData/edrd/gpm_gmi/2014-06-23/'
    LogFile='log/scenedump_logFile.dat'
  /
\end{verbatim}
The code can be invoked from the command line as follows: 
\begin{verbatim}
 $ bin/sceneDump < scene_dump.nl
\end{verbatim}
Once the code has finished, a user can point to the new dumped scene files and rerun the COAT.  The following example 
uses NWP GFS collocated scene files as input to {\tt sceneDump}.
\begin{verbatim}
  $ ls -d -1 ${path-to-data}/edrd/gpm_gmi/2014-06-23/*GFS* > scene.list
  $ idl
  IDL--> !PATH = !PATH + ':src/idl/coyote/'
  IDL-->.r main_AT
  Choose instrument:
  ...
  Choose Scene data source
  0 - New Scene dump file
  1 - Old Scene file
  : 0
\end{verbatim}

%\section{Miscellaneous}
%First thing you'll be working on is a GOES BUFR to FMSDR writer.  
%See {\tt COAT/src/bufr/} for examples.  These routines link to NCEP BUFR libraries for reading BUFR ({\tt src/bufr/bufrlib/}) and MIRS modules for FMSDR writing 
%{\tt src/mirs/IO\_MeasurData.f90}).
%Examples of reading BUFR and conversion into FMSDR format are in: 

%\begin{verbatim}
%  src/bufr/amsr2_bufr/
%  src/bufr/gmi_bufr/
%  src/bufr/ssmis_bufr/
%  ...
%\end{verbatim}
\begin{appendices}
\chapter{Running Scatterometers through the COAT}
\section{IDL COAT basics for Scatterometers}
See Appendix~\ref{chpt:AMVCOAT}.  You'll need to add configuration for the 
sensor in {\tt config/configParam\_SAT\_SENSOR.pro}, 
{\tt importConfigParam.pro}, the 
sensor to the user-selectable list in and also a link 
to the AMV assessment GOTO statement in {\tt main\_AT.pro}.  
~~\\~~\\
Examples of IDL 
configuration for ISS\_RapidScat are in {\tt config/configParam\_ISS\_RapidScat.pro}.
\section{Fortran COAT basics for Scatterometers}

For surface wind scatterometers, the COAT uses the atmospheric motion vector 
Fortran and IDL types described in Appendix~\ref{chpt:AMVCOAT}.  
{\bf Under construction...}

\chapter{Running AMV through the COAT}\label{chpt:AMVCOAT}
\section{IDL COAT basics for AMV}
\subsection{IDL COAT input for AMV - user defined parameters}
User defined parameters for AMV data are located in {\tt config/configParam\_AMV.pro}.
\begin{verbatim}
   $ cd config/
   $ cat configParam_AMV.pro
   ; Set map range to plot
   paramStruct.MIN_LAT = -90
   paramStruct.MAX_LAT = 90
   paramStruct.MIN_LON = -180

   paramStruct.sensorName    = 'AMV'
   paramStruct.sensorID      = 'AMV'
   paramStruct.date          = '20140401_05'

   ; AMVScene files
   paramStruct.AMVsceneListFile = 'amvscene00z.list'
   ;---for ECMWF collocated NWP winds to AMV locations/pressures
   paramStruct.NWPsceneListFile = 'nwpscene00z.list.ecm'

   ; unique options for AMV
   amvOptions = {figPath: './images/amv/', $  ; output path for figures
                 nwpStr: 'ECMWF', $           ; NWP string for labels
                 TimeStr: '00', $             ; cycle time
                 figStr: '', $                ; dummy
                 date: paramStruct.date, $    ; date string from above
                 pngOrderflag: 1, $           ; IDL flag to order png files
                 psorpng: 1, $                ; ps=1, png=0
                 qualityMark: 0, $            ; apply GSI quality flag QC
                 gsiblacklist: 0, $           ; use GSI blacklisting 
                 highlevel: 0, $              ; plot winds p<500hPa
                 lowlevel: 0, $               ; plot winds p>=500hPa
                 ecmwfqi: 0}                  ; use ECMWF QI settings

   ;---whatever floats your boat
   amvOptions.figStr = paramStruct.date + '_' + $
                       amvOptions.TimeStr + '_' + amvOptions.nwpStr ;  + 'qcmbl'

   ;---copy amvOptions structure to paramStruct structure
   paramStruct.amvOptions    = amvOptions
\end{verbatim}  
\subsection{Running the IDL COAT for AMV}
Running the COAT for AMV/SATWND data in IDL is similar to running radiance/BT data sets.  
\begin{verbatim}
  $ cd COAT
  $ mkdir images/amv/
  $ idl
  IDL--> !PATH = !PATH + ':src/idl/coyote/'
  IDL-->.r main_AT
  Choose instrument:
  ...
  18 : SATWND AMV
  : 18
  Read AMV data again?
  1 - YES
  2 - NO, to plot unfiltered data
  3 - NO, to plot filtered GSI QC'd blacklisted
  4 - NO, to plot low level filtered
  5 - NO, to plot high level filtered
  : 1
\end{verbatim}
\subsection{IDL COAT output for AMV}
Based on the options selected in {\tt config/configParam\_AMV.pro}, the COAT either
produces output .eps for .png files.  These are then concatenated using 
pdf\LaTeX into two data assessment reports or DARs for the AMV data :
\begin{itemize}
  \item {\tt AMV\_DAR\_maps.pdf}
  \item {\tt AMV\_DAR\_stratified.pdf}
\end{itemize}
\section{Fortran COAT basics for AMV}
\subsection{SATWND BUFR Converter}
{\tt bin/AMVBUFR\_to\_AMVScene} is a namelist driven executable that reads NCEP BUFR SATWND data and converts it to MIRS {\it sceneAMV} format
 (see {\tt src/lib/IO\_SceneAMV.f90} and/or \\
{\tt src/idl/mirs\_idl/io\_sceneamv.pro}).  An example namelist for the BUFR to {\it sceneAMV} converter 
is given below.
\begin{verbatim}
  $ cat bufr_satwnddump.nl
  &ControlBUFRdump
     ! list of GDAS SATWND BUFR data files (full path)
     BUFRFilesList='bufr_gdas_2014-04-14.list'
     ! GSI Error table for conventional observations (including SATWNDS)
     ErrorTableFile='prepobs_errtable.global'
     ! output path for binary sceneAMV data
     pathBinout='data/TestbedData/amv/2014-04-14/'
  /
\end{verbatim}
From the command line, the routine can be run as follows.
\begin{verbatim}
  $ bin/AMVBUFR_to_AMVScene < bufr_satwnddump.nl 
\end{verbatim}
\subsection{NWP Collocation}
NWP collocation/interpolation to AMV latitude, longitude and pressure is performed in the routine {\tt bin/colocNWPsatwind} in the COAT main directory.
\begin{verbatim}
  $ ls -d -1 data/TestbedData/amv/2014-04-14/*ecmwf* > satwnd_windNWP_2014-04-01.list_ecmwf
  $ cat coloc_NWP_AMV.CONFIG
   &ControlNWP
     ! list of SATWND converted BUFR files
     WindFileList=OBS_WIND_LIST    ! 'satwnd_windNWP_2014-04-01.list_ecmwf'
     atmNWPFilesList='NWP_ATM_LIST'
     sfcNWPFilesList='NWP_SFC_LIST'
     windNWPFilesList='NWP_WIND_LIST'
     pathNWPout='EDR_PATH/DATE/'
     CovBkgFileAtm='../data/static/CovBkgStats/CovBkgMatrxTotAtm_all.dat'
     norbits2process=1000000
     LogFile='LOG_PATH/amvlog.dat'
     nprofs2process=1000000
     sensor_id=6
     nwp_source=NWPID
   /
\end{verbatim}
Simliar to the examples given in Section~\ref{ssec:NWPColoc}, the variables in all 
capitals (OBS\_WIND\_LIST, NWP\_ATM\_LIST, NWP\_SFC\_LIST, NWP\_WIND\_LIST, EDR\_PATH, DATE, 
LOG\_PATH, NWPID) are set in the script {\tt generate\_coloc\_config.bash} and imported from
{\tt paths.setup} and {\tt instrument.setup}.  

\end{appendices}

\end{document}

